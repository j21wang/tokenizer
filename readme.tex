\documentclass{article}

\title{CS 214 Assignment 1: Tokenizer}
\author{Kevin Sung and Joyce Wang}

\begin{document}

\maketitle

\section{Implementation}

\subsection{struct \emph{TokenizerT\_}}

The struct contains 5 members:
\begin{itemize}
    \item \emph{separators}: the string of separators
    \item \emph{tokens}: the string of tokens
    \item \emph{firstIndex} and \emph{secondIndex}: indices used to parse tokens
    \item \emph{arr}: bit array of separator chars
\end{itemize}

\subsection{TokenizerT *\emph{TKCreate}}

When a \emph{TokenizerT} is created, the input "separator" and "token" strings are
copied into \emph{separators} and \emph{respectively}, and \emph{firstIndex} and
\emph{secondIndex} are both set to 0. \emph{arr}is set to an array of 256 short ints,
where \emph{arr}[\emph{c}] = 1 if \emph{c} is a separator character, 0 otherwise.
This bit array allows us to check whether a given character is a separator character in
constant time.

\subsection{char *\emph{TKGetNextToken}}
First, the indices \emph{first} and \emph{second} are advanced to the next non-separator
character. Then, \emph{second} is advanced to the following separator. The size of the
token is equal to the number of characters between \emph{first} and \emph{second},
including the character at \emph{first} but not \emph{second}. A string of the correct
size is created, and \emph{first} is advanced to \emph{second} while the characters that
\emph{first} passes by are added to the string. Once \emph{first} has reached \emph{second},
the string is terminated and returned.

\subsection{void \emph{TKDestroy}}
This method just frees the memory allocated to each member of the tokenizer, and then the
tokenizer itself.

\section{Time Complexity}

I will analyze the method \emph{TKGetNextToken} and count the number of character comparisons.
Each character in the input "token" string is traversed twice, once by \emph{first} and once
by \emph{second}. Each time a character is encountered, it is checked to see whether it is a
separator or not. This check is done in constant time due to the bit array of separator
characters. In the worst case, the character is a special character and must be resolved, but
this is done with no more than a constant number of comparisons (with the list of special
characters). Therefore, for each character in token, no more than a constant number of
comparisons are performed. The running time does not depend on the length of the "separator"
string due to the use of the bit array, which acts as a hash table. Therefore, the method
runs in \emph{O(n)} time, where \emph{n} is the number of characters in the "token" string.

\end{document}
